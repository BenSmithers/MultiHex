\documentclass[12pt,a4paper]{article}

\usepackage[margin=0.75in]{geometry}

%Used for doing math things
\usepackage{amsmath}

%Used for fonts, I guess?
\usepackage{amsfonts}

%allows for bold-faced mathematical symbols (like beta)
\usepackage{bm}


%I have no idea what this does, so I commented it out. If the file breaks, try un-commenting this out.
%\usepackage[latin1]{inputenc}

% Amssymb is for assorted symbols
\usepackage{amssymb}

%This allows for the enumeration of
\usepackage{enumitem}

%Used for seting the spacing density. 
\usepackage{setspace}

%The first package is used for incorporating images into the LaTeX file
\usepackage{graphicx}

%Allows the usage of the [H] argument, which forces figure/tabular placement to be HERE and nowhere else. 
\usepackage{float}

%These are useful commands I've hard-coded in.  
\newcommand{\eps}{\epsilon}
\newcommand{\braket}[1]{\left<#1\right>}
\newcommand{\bra}[1]{\left< #1 \right|}
\newcommand{\ket}[1]{\left| #1 \right>}
\newcommand{\comm}[1]{\left[ #1 \right]}
\newcommand{\cross}{\times}
\newcommand{\abs}[1]{\left| #1 \right|}
\newcommand{\bvec}[1]{\bm{#1}}

%rcurs is the basic cursive one
%\def\rcurs{{\mbox{$\resizebox{.16in}{.08in}{\includegraphics{ScriptR}}$}}}
%\def\brcurs{{\mbox{$\resizebox{.16in}{.08in}{\includegraphics{BoldR}}$}}}
%\def\hrcurs{{\mbox{$\hat \brcurs$}}}



%Use this if you use images. Default path is the file's directory. 

%\graphicspath{ {Path} }



% Fancy Title

\usepackage{array}
\newcolumntype{P}[1]{>{\centering\arraybackslash}p{#1}}
\newcolumntype{R}[1]{>{\raggedright\arraybackslash}p{#1}}
\newcolumntype{L}[1]{>{\raggedleft\arraybackslash}p{#1}}
\newcommand{\Qbox}[1]{\noindent\fbox{\parbox{\textwidth}{#1}}}

\singlespacing

\begin{document}
\begin{table}
	\centering
	\begin{tabular}{R{2in} P{2in} L{2in}}
		Benjamin Smithers & \begin{tabular}{c}{\Large MultiHex} \\ {\Large The Math}  \end{tabular} & February 2020 \\\hline
	\end{tabular}

\end{table}

\section{Sunlight Algorithm}
\subsection{Mathematical Framework}
We first define constants
\begin{itemize}
	\item \(t\): the present time [min]
	\item \((l_{a},l_{o})\): latitude and longitude of location [rad]
	\item \(\xi\): axial tilt of rotation 
	\item \(\omega\): frequency of days
	\item \(\gamma\): frequency of years
\end{itemize}
Note that the frequency of days will need to be slightly adjusted 

With these, we parametrize the position of the Earth relative to the sun with respect to time 
\begin{equation}\label{eq:sun_earth}
\vec{P}_{e} = \left( \begin{array}{ccc} \cos\gamma t & \sin\gamma t & 0 \end{array}\right).
\end{equation}
Suppose that at \(t=0\), we say the sun is tilted away from the sun, towards the \(\hat{x}\)-axis. At this time, \(l_{o}\) is also pointed along the \(\hat{x}\)-axis. \\

We construct a unit-vector from the Earth's core to a position on the surface, before applying the tilt
\begin{equation}\label{eq:basic}
\vec{P}_{s}' = \left(\begin{array}{ccc}\cos l_{a}\cos l_{d} & \cos l_{a}\sin l_{d} & \sin l_{a} \end{array}\right)
\end{equation}

Now we rotate Eq~\eqref{eq:basic} about the \(\hat{z}\)-axis to account for the planet's spin and its rotation around the star.
\begin{align}
\vec{P}_{s}'' &=\left(\begin{array}{ccc} \cos(\omega + \gamma)t & -\sin(\omega + \gamma) t & 0 \\ \sin(\omega+\gamma) t & \cos(\omega + \gamma) t & 0 \\ 0 & 0 & 1 \end{array} \right) \left(\begin{array}{c}\cos l_{a}\cos l_{d} \\ \cos l_{a}\sin l_{d} \\ \sin l_{a} \end{array}\right) \\
\vec{P}_{s}'' &=  \left(\begin{array}{c} \cos(\omega+\gamma)t\cos l_{a}\cos l_{d} - \sin(\omega+\gamma)t\cos l_{a}\sin l_{d} \\ \sin(\omega + \gamma)t\cos l_{a}\cos l_{d} + \cos(\omega+\gamma)t\cos l_{a}\sin l_{d} \\ \sin l_{a} \end{array}\right) \label{eq:rotated}
\end{align}

For instance, for a planet without spin \(\omega = 0\). There is rotation, relative to the star, as the planet goes around the sun. The planet and star are tidally locked when \(\omega=\gamma\). Now Eq~\eqref{eq:rotated} is rotated about the \(\hat{y}\)-axis to incorporate axial tilt. 
\begin{align}
\vec{P}_{s} &= \left(\begin{array}{ccc}\cos\xi  & 0 & -\sin\xi \\ 0 & 1 & 0 \\ \sin\xi & 0 & \cos\xi \end{array}\right)\left(\begin{array}{c} \cos(\omega+\gamma)t\cos l_{a}\cos l_{d} - \sin(\omega+\gamma)t\cos l_{a}\sin l_{d} \\ \sin(\omega + \gamma)t\cos l_{a}\cos l_{d} + \cos(\omega+\gamma)t\cos l_{a}\sin l_{d} \\ \sin l_{a} \end{array}\right) \\
\vec{P}_{s} &= \left(\begin{array}{c} \cos(\omega+\gamma)t\cos l_{a}\cos l_{d}\cos\xi - \sin(\omega+\gamma)t\cos l_{a}\sin l_{d}\cos\xi -\sin\xi\sin l_{a} \\ \sin(\omega + \gamma)t\cos l_{a}\cos l_{d} + \cos(\omega+\gamma)t\cos l_{a}\sin l_{d} \\ \cos(\omega+\gamma)t\cos l_{a}\cos l_{d}\sin\xi - \sin(\omega+\gamma)t\cos l_{a}\sin l_{d}\sin\xi + \cos\xi\sin l_{a} \end{array}\right) \label{eq:dir}
\end{align}
Using Equations~\eqref{eq:sun_earth}-\eqref{eq:dir}, we can define a `light level'
\begin{equation}
L = -(\vec{P}_{e}\cdot \vec{P}_{s}).
\end{equation}
And interpret it as such
\begin{itemize}
	\item \(L < -0.1\): nighttime
	\item \(-0.1 < L < 0.1\): twilight hours
	\item \(L>0.1\): daytime 
\end{itemize}
With \(+1\) correlating with the sun directly overhead. 

\subsection{Implementation}
\begin{itemize}
	\item Finding sunrise and sunset: cache evaluated trigonometric values and propagate time-dependent ones forward until a point is found with a zero-valued light level. 
\end{itemize}

\end{document}

