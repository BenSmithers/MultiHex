\documentclass[12pt,a4paper]{article}

\usepackage[margin=0.75in]{geometry}

%Used for doing math things
\usepackage{amsmath}

%Used for fonts, I guess?
\usepackage{amsfonts}

%allows for bold-faced mathematical symbols (like beta)
\usepackage{bm}


%I have no idea what this does, so I commented it out. If the file breaks, try un-commenting this out.
%\usepackage[latin1]{inputenc}

% Amssymb is for assorted symbols
\usepackage{amssymb}

%This allows for the enumeration of
\usepackage{enumitem}

%Used for seting the spacing density. 
\usepackage{setspace}

%The first package is used for incorporating images into the LaTeX file
\usepackage{graphicx}

%Allows the usage of the [H] argument, which forces figure/tabular placement to be HERE and nowhere else. 
\usepackage{float}

%These are useful commands I've hard-coded in.  
\newcommand{\eps}{\epsilon}
\newcommand{\braket}[1]{\left<#1\right>}
\newcommand{\bra}[1]{\left< #1 \right|}
\newcommand{\ket}[1]{\left| #1 \right>}
\newcommand{\comm}[1]{\left[ #1 \right]}
\newcommand{\cross}{\times}
\newcommand{\abs}[1]{\left| #1 \right|}
\newcommand{\bvec}[1]{\bm{#1}}

%rcurs is the basic cursive one
%\def\rcurs{{\mbox{$\resizebox{.16in}{.08in}{\includegraphics{ScriptR}}$}}}
%\def\brcurs{{\mbox{$\resizebox{.16in}{.08in}{\includegraphics{BoldR}}$}}}
%\def\hrcurs{{\mbox{$\hat \brcurs$}}}



%Use this if you use images. Default path is the file's directory. 

%\graphicspath{ {Path} }



% Fancy Title

\usepackage{array}
\newcolumntype{P}[1]{>{\centering\arraybackslash}p{#1}}
\newcolumntype{R}[1]{>{\raggedright\arraybackslash}p{#1}}
\newcolumntype{L}[1]{>{\raggedleft\arraybackslash}p{#1}}
\newcommand{\Qbox}[1]{\noindent\fbox{\parbox{\textwidth}{#1}}}

\singlespacing

\begin{document}
\begin{table}
	\centering
	\begin{tabular}{R{2in} P{2in} L{2in}}
		Benjamin Smithers & \begin{tabular}{c}{\Large MultiHex} \\ {\Large The Math}  \end{tabular} & February 2020 \\\hline
	\end{tabular}

\end{table}

\section{Sunlight Algorithm}

\subsection{Background Mathematics}

This algorithm is used to approximately find the sunrise and sunset on a given day for a given latitude. We define:
\begin{itemize}
	\item \(\theta\): correlates with the position on the map's local time. \(\theta=0\) opposite the sun (midnight) and grows clockwise while looking from above the North pole (Easterly spin). 
	\item \(\xi\): latitude of location, or radians from the equator. 
	\item \(\gamma\): axial tilt of planet
	\item \(\alpha\): correlates with the time of the year. \(\alpha=0\) is winter solstice, \(\alpha=\pi/2\) is vernal equinox, and so on.
\end{itemize}
Consider a sphere of radius \(R\) representing a planet, centered at the origin, and bisected by a plane at \(x=0\). We consider one half day and one half night. \\

Now consider a single point on that sphere. If we rotate the sphere one full revolution about the \(\hat{z}\) axis, the point will trace out a path parametrized by
\begin{equation}
\vec{P} = \left(\begin{array}{ccc} R\cos\theta\sin\xi & R\sin\theta\sin\xi & R\cos\xi \end{array}\right).
\end{equation}
We want to find the points where the aforementioned plane intersects this path after applying transformations to account for axial tilt and seasonal rotations. \\

First, we apply the axial tilt. We rotate the planet, and therefore the parametrization, about the \(\hat{x}\)-axis by an angle \(\gamma\).  
\begin{align}
\vec{P}'&= \left(\begin{array}{ccc} 1 & 0 & 0\\ 0 & \cos\gamma & -\sin\gamma \\ 0 & \sin\gamma & \cos\gamma \end{array}\right)\left(\begin{array}{c} R\cos\theta\sin\xi \\ R\sin\theta\sin\xi \\ R\cos\xi \end{array}\right) \\
\vec{P}'&=\left(\begin{array}{c} R\cos\theta\sin\xi \\ R\sin\theta\cos\gamma\sin\xi - R\sin\gamma\cos\xi \\ R\sin\theta\sin\gamma\sin\xi + R\cos\gamma\cos\xi  \end{array}\right) \\
\end{align}
Now, we rotate this parametrization about the \(\hat{z}\)-axis by an angle \(\alpha\) to apply seasonal variations.
\begin{align}
\vec{P}''&= \left(\begin{array}{ccc} \cos\alpha & -\sin\alpha & 0 \\ \sin\alpha & \cos\alpha & 0 \\ 0 & 0 & 1 \end{array}\right)\left(\begin{array}{c} R\cos\theta\sin\xi \\ R\sin\theta\cos\gamma\sin\xi - R\sin\gamma\cos\xi \\ R\sin\theta\sin\gamma\sin\xi + R\cos\gamma\cos\xi  \end{array}\right) \\
\vec{P}''&= \left(\begin{array}{c} R\cos\theta\cos\alpha\sin\xi -R \sin\theta\sin\alpha\cos\gamma\sin\xi + R\sin\alpha\sin\gamma\cos\xi \\ \left[\ldots\right] \\\left[\ldots\right] \end{array}\right) 
\end{align}
Intersections of this parametrization with the plane at \(x=0\) are independent of the values of \(y\) and \(z\), so we can skip those calculations. Intersections then simply satisfy the relation
\begin{equation}
R\cos\theta\cos\alpha\sin\xi -R \sin\theta\sin\alpha\cos\gamma\sin\xi + R\sin\alpha\sin\gamma\cos\xi = 0
\end{equation}
To simplify representation, we divide by \(R\) and define
\begin{align}\label{eq:const}
A&= \cos\alpha\sin\xi, & B&= \sin\alpha\cos\gamma\sin\xi, & &\text{and} & C&=\sin\alpha\sin\gamma\cos\xi.
\end{align}
So that we have
\begin{equation}
A\cos\theta - B\sin\theta + C = 0.
\end{equation}
We solve for \(\sin\theta\) from here.
\begin{align}
A\cos\theta  &=  B\sin\theta - C \\
A^{2}\cos^{2}\theta  &=  B^{2}\sin^{2}\theta + C^{2} - 2CB\sin\theta \\
A^{2} - A^{2}\sin^{2}\theta  &=  B^{2}\sin^{2}\theta + C^{2} - 2CB\sin\theta \\
0&=  (A^{2} + B^{2})\sin^{2}\theta - 2CB\sin\theta  + (C^{2}-A^{2})
\end{align}
It is trivially easy to solve such an equation for \(\sin\theta\)
\begin{equation}
\sin\theta = \dfrac{2CB \pm \sqrt{4C^{2}B^{2} + (A^{2}+B^{2})(A^{2}-C^{2}) }}{2(A^{2}+B^{2})},
\end{equation}
and therefore calculate \(\theta\), and derive its corresponding time.  \\[10pt]

Of course, this does not \textit{always} have solutions. When 
\begin{equation}\label{eq:test}
4C^{2}B^{2} + (A^{2}+B^{2})(A^{2}-C^{2})  < 0
\end{equation}
there is no solution: suggesting the point lies within an arctic circle at the wrong time of the year.


\subsection{Implementation}
First, calculate \(\alpha\) for the given day of the year. Knowing \(\gamma\) and \(\xi\), calculate the constants \(A\), \(B\), and \(C\) according to Equation~\eqref{eq:const}.\\

First the inequality of Equation~\eqref{eq:test} should be tested. If the inequality is true, use \(\alpha\) and \(\xi\) to determine whether it is night or day. 
\begin{center}
If \(\xi>0\) and \(\tfrac{\pi}{2} < \alpha < \tfrac{3\pi}{2}\), then it's daytime. Nighttime otherwise. 
\end{center}
Otherwise get the two solutions to \(\sin\theta\). Use arcsine to get the associated values of \(\theta\); this will give angles in the range \(\left(-\pi,\pi\right)\). 
\begin{itemize}
	\item If \(\tfrac{\pi}{2}<\alpha<\tfrac{3\pi}{2}\) (Summer in North, Winter in South)
	\begin{itemize}
		\item If \(\xi > 0\). Values accurate; value under 0 is sunset. 
		\item If \(\xi < 0\). Change both values \(\theta\to\pi-\theta\). Larger of the two is sunset. 
	\end{itemize}
	\item If \(\alpha < \tfrac{\pi}{2}\) or \(\alpha>\tfrac{3\pi}{2}\) (Winter in North, Summer in South)
	\begin{itemize}
		\item If \(\xi > 0\). Change both values \(\theta\to\pi-\theta\). Larger of the two is sunset. 
		\item If \(\xi < 0\). Values accurate; value under 0 is sunset. 
	\end{itemize}
	
\end{itemize}



\end{document}

